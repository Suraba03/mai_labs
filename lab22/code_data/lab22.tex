\documentclass[a4paper,12pt]{article}
\usepackage{ upgreek }
\usepackage{ tipa }
\usepackage[T2A]{fontenc}
\usepackage[utf8]{inputenc}
\usepackage[english,russian]{babel}
\usepackage{amsmath,amsfonts,amssymb,amsthm,mathtools}

\begin{document}

\subsection*{13.3 Формулы Тейлора\\для основных элементарных функций}

\quad1. $f(x) = \sin x$. Функция $\sin x$ обладает производными всех порядков. Найдем для нее формулу Тейлора при $x_0=0$, т.е. формулу Маклорена (13.8). Было доказано (см. п. 10.1), что $(\sin x)^{(m)}=\sin (x+m\frac{\pi}{2})$, поэтому 

\[
    f^{(m)}(0)=\sin \frac{m\pi}{2}= 
    \left\{
        \begin{array}{rcl}
            0 \hspace{15mm} \text{для } m = 2k, \\
            (-1)^{k} \hspace{5mm} \text{для } m = 2k + 1,\\
        \end{array}
    \right.
    k = 0, 1, 2, ... , \eqno(13.16)
\] \\

и, согласно формуле (13.5), \\

\[
    \sin x = x - \frac{x^3}{3!} + \frac{x^5}{5!} - \frac{x^7}{7!} + ... +
    (-1)^n \frac{x^{2n + 1}}{(2n + 1)!} + o(x^{2n + 2})
\],\\

при $x \longrightarrow 0, n=0, 1, 2, ... ,$ или, короче,
\[
    \sin x = \sum_{k = 0}^n (-1)^k \frac{x^{2k + 1}}{(2k + 1)!} + o(x^{2n + 2})
    \text{ при } x \longrightarrow 0.
\]
Мы записали здесь остаточный член в виде $o(x^{2n + 2})$, а не в виде $o(x^{2n + 1})$,
так как следующий за последним выписанным слагаемым член многочлена Тейлора, в силу 
(13.16), равен нулю.

2. $ f(x) = \cos x $. Как известно (см. п. 10.1),
\[
    f^{(m)} (x) = \cos (x + \frac{m\pi}{2}),
\]
поэтому \\
\[
    f^{(m)} (0) = \cos \frac{m\pi}{2} = 
    \left\{
        \begin{array}{rcl}
            0& \hspace{7mm} \text{для } m = 2k + 1, \\
            (-1)^{k}& \text{для } m = 2k,\\
        \end{array}
    \right.
    k = 0, 1, 2, ... ,
\]
и 
\[
    \cos x = 1 - \frac{x^2}{2!} + \frac{x^4}{4!} - \frac{x^6}{6!} + ... +
    (-1)^n \frac{x^{2n}}{(2n)!} + o(x^{2n + 1}),
\]
при $x \longrightarrow 0 $, или, короче,
\[
    \cos x = \sum_{k = 0}^n (-1)^k \frac{x^{2k}}{(2k)!} + o(x^{2n + 1})
\]
при $x \longrightarrow 0, n = 0, 1, 2, ... .$

3. $ f(x) = e^x $. Так как $ (e^x)^{(n)} = e^x $, то $ f^{(n)} (0) = 1, n = 0, 1, ... , $ \\ следовательно, 
\[
    e^x = 1 + x + \frac{x^2}{2!} + \frac{x^3}{3!} + \frac{x^4}{4!} + ... +
    \frac{x^{n}}{n!} + o(x^{n}), \eqno(13.17)
\]
при $ x \longrightarrow 0, n=0, 1, 2, ... , $ или, короче,
\[
    e^x = \sum_{k = 0}^n \frac{x^k}{k!} + o(x^n) \text{ при }
    x \longrightarrow 0. 
\]
Отсюда, заменив x через -x, получим 
\[
    e^{-x} = \sum_{k = 0}^n (-1)^k \frac{x^k}{k!} + o(x^n) \eqno(13.18)
\]
при $x \longrightarrow 0, n = 0, 1, 2, ... .$

4. $ \sh x = \frac{e^x - e^{-x}}{2} $ и $ \ch x = \frac{e^x + e^{-x}}{2} $. Сложив и вычтя (13.17) и (13.18), при $ x \longrightarrow 0, n=0, 1, 2, ... $ будем иметь 
\[
    \sh x = \sum_{k = 0}^n \frac{x^{2k + 1}}{(2k + 1)!} + o(x^{2n + 2}), 
    \ch x = \sum_{k = 0}^n \frac{x^{2k}}{(2k)!} + o(x^{2n + 1}).
\]

В силу единственности представления функции  в указанном виде (см. п. 13.2), полученные соотношения являются формулами Тейлора для функций $ \sh x $ и $ \ch x $.

5. $ f(x) = (1 + x)^\alpha $, где $\alpha$ -- некоторое фиксированное число, а 
$ x \textgreater -1 $. Так как 
\[
    f^{(n)} (x) = \alpha(\alpha - 1) ... (\alpha - n + 1)(1 + x)^{\alpha - n},
\]
то
\[
    f^{(n)} (0) = \alpha(\alpha - 1) ... (\alpha - n + 1),
\]
следовательно,
\[
    (1 + x)^{\alpha} = 1 + \alpha x + \frac{\alpha(\alpha - 1)}{2} x^2 
    + \frac{\alpha(\alpha - 1)(\alpha - 2)}{3!} x^3 + ... + 
    \frac{\alpha(\alpha - 1)...(\alpha - n + 1)}{n!} x^n + o(x^n)
\]
при $ x \longrightarrow 0 $, или, короче,
\[
    (1 + x)^{\alpha} = 1 + \sum_{k = 1}^n \frac{\alpha(\alpha - 1)...
    (\alpha - k + 1)}{k!} x^k + o(x^n)
    \text{ при } x \longrightarrow 0, n = 1, 2, ... . \eqno(13.19)  
\]

Если $ \alpha = n $ -- неотрицательное целое $ (n = 0, 1, 2, ...) $, то функция 
$ f(x) = (1 + x)^n $ является многочленом степени $n$, и имеет место тождество
\[
    f(x) = Q_n(x) + o((x - x_0)^n), x \longrightarrow x_0, 
\]

\end{document}

